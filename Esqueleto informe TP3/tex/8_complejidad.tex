\section{Conclusiones}

En conclusión, este informe abordó el problema de planificar el entrenamiento de la selección de fútbol de Argentina de manera óptima, maximizando la ganancia total a lo largo de un período de tiempo dado. Se formuló una función de recurrencia que describe la mejor manera de tomar decisiones sobre cuándo entrenar y cuándo descansar, considerando el esfuerzo requerido y la energía disponible para cada día. 

Se demostró que la complejidad temporal del algoritmo propuesto es cuadrática en función de la longitud de las entradas de esfuerzo y energía, lo que significa que su tiempo de ejecución se encuentra acotado por una función polinómica. Esto es una buena noticia, ya que indica que el algoritmo es eficiente y escalable en la práctica.

Además, se realizaron mediciones empíricas que respaldaron la complejidad teórica, mostrando que el tiempo de ejecución del algoritmo se comporta de manera consistente con una complejidad cuadrática en la longitud de las entradas. No se encontraron indicios de un comportamiento pseudopolinomial, lo que refuerza la validez de la cota temporal.

En resumen, este informe proporciona una solución eficiente y escalable para el problema de planificar el entrenamiento de la selección de fútbol de Argentina, respaldada por análisis teóricos y mediciones empíricas. Esto debería permitir que el equipo técnico tome decisiones informadas sobre cómo estructurar el entrenamiento para maximizar las posibilidades de éxito en el próximo Mundial.