\section{Codigo solución *}

En un principio se agrega la implementación de la función esfuerzo la cual es vital para poder solucionar el problema mediante la tecnica de programación dínamica:


\begin{lstlisting}[language=Python]
def funcion_esfuerzo(i: int, j: int, esfuerzo: list, energia: list):
    return energia[j] if esfuerzo[i] > energia[j] else esfuerzo[i]
\end{lstlisting}

Una vez habiendo construido la ecuación de recurrencia se puede construir el codigo solución del problema utilizando la tecnica button-up para evitar llamados de recurrencias inecesarios que realenticen el problema

\begin{lstlisting}[language=Python]
def explorar_espacion_solucion(esfuerzo: list, energia: list):
    n = len(esfuerzo)
    espacio_solucion = [[0 for _ in range(n)] for _ in range(n)]

    espacio_solucion[0][0] = funcion_esfuerzo(0, 0, esfuerzo, energia)
    espacio_solucion[1][0] = funcion_esfuerzo(1,0, esfuerzo, energia)
    espacio_solucion[1][1] = max(funcion_esfuerzo(1,1, esfuerzo, energia) + espacio_solucion[0][0], espacio_solucion[1][0])

    for i in range(2, n):
        for j in range(i + 1):
            if j == 0:
                espacio_solucion[i][0] = espacio_solucion[i-2][i-2] + funcion_esfuerzo(i, 0, esfuerzo, energia)
            else:
                espacio_solucion[i][j] = max([espacio_solucion[i-1][j-1] + funcion_esfuerzo(i, j, esfuerzo, energia), espacio_solucion[i][j-1], espacio_solucion[i-2][j-2] + funcion_esfuerzo(i,0, esfuerzo, energia)])
    return espacio_solucion, espacio_solucion[n-1][n-1]
\end{lstlisting}

El codigo anteriormente dado contiene la solución por programación dínamica y haciendo uso de la memoización del metodo, se aprovecha tambien retornar la matriz con la exploración de las posibles respuestas de manera que sea posible realizar una reconstrucción en la secuencia de entrenamiento para poder saber que hacer cada día

Por otro lado, se agrega el codigo solución de la reconstrucción de la secuencia de entrenamiento, el cual será explicado en la siguiente sección del informe

\begin{lstlisting}[language=Python]
def _encontrar_secuencia(espacio_solucion: list, secuencia: str, n: int):
    if n < 0:
        return
    if n == 0 and len(secuencia) < len(espacio_solucion):
        secuencia.append("Entreno")
        return
    indice = espacio_solucion[n].index(max(espacio_solucion[n]))
    indice += 1
    for _ in range(indice):
        secuencia.append("Entreno")
    if len(secuencia) < len(espacio_solucion):
        secuencia.append("Descanso")
    n -= indice + 1
    return _encontrar_secuencia(espacio_solucion, secuencia, n)
 
def encontrar_secuencia_entrenamiento(espacio_solucion: list):
    secuencia = []
    _encontrar_secuencia(espacio_solucion, secuencia, len(espacio_solucion) - 1)
    secuencia.reverse()
    return secuencia
\end{lstlisting}