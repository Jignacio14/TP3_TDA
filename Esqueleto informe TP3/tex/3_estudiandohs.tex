\section{Estudiando Hitting Set Problem}

\subsection{Demostrando su pertenencia a NP}

Para demostrar que Hitting Set se encuentra en NP bastaría con tener una solución del problema: un conjunto $H$ y y el set de datos de entrada al mismo, un conjunto de subconjuntos $B = B_1, B_2, ... B_m$ ($B_i \subseteq A \forall i$) de manera que, simplemente, se debe iterar sobre el conjunto de subconjuntos y sobre cada subconjunto validando si alguno de los elementos $b_j \in B_i$ ciertamente pertenece a $H$. En caso de haber probado con todos los elementos de un $B_i$ y ninguno de sus elementos $b_j$ pertenezca a $H$ entonces, la solución brindada no es un Hitting Set, en caso de haber llegado al final entonces dicho conjunto si es un Hitting Set

\subsubsection{Implementación del validador}

\begin{lstlisting}[language=Python, caption= validador, label=python_code]
def is_solution(hitting_set, subsets):
    for subset in subsets:
        flag = False
        for element in subset:
            if element in hitting_set: 
                flag = True
                break
        if not flag: return False 
    return True
\end{lstlisting}

\subsubsection{Complejidad temporal del validador}

Para verificar finalmente que dicho problema falta con probar que este validador posee tiempo polinomial

Veamos que: 

\begin{itemize}
    \item Recorrer la lista de subconjuntos es lineal en la cantidad de subconjuntos, dicha cantidad se llamará $m$.
    \item Recorrer los elementos del subconjunto $B_i$ es lineal en la cantidad de elementos del subconjunto, dicha cantidad se llamará $k$.
\end{itemize}

Ahora si decimos generalizamos $k$ como el máximo cardinal existente de algún conjunto $B_i$ se puede establecer la siguiente cota:

$$
    T(n) = O(m \times k)
$$

Como dicha complejidad es polinomial a $m$ y $k$ entonces se cumple que Hitting Set esta  en NP

\subsection{Demostrando pertenece a NP completo}

Para probar este segmento del informe se debe realizar una reduccion de algun problema NP-Completo conocido a Hitting-Set, en este caso, dicho problema sera Vertex Cover.

En un principio se establecen ambos problemas de decisión: 

\begin{center}
  \begin{minipage}{0.8\textwidth}
    \fbox{%
      \parbox{\dimexpr\linewidth-2\fboxsep-2\fboxrule}{%
        \textbf{El problema de decisión del vertex cover implica, dado un grafo G y un número k, determinar si existe un vertex cover de tamaño a lo sumo k \\}
      }%
    }
  \end{minipage}
\end{center}

Por otro lado: 

\begin{center}
  \begin{minipage}{0.8\textwidth}
    \fbox{%
      \parbox{\dimexpr\linewidth-2\fboxsep-2\fboxrule}{%
        \textbf{El problema de decisión del hitting set implica, sea A un conjunto, B un conjunto de subconjuntos pertenecientes a A, determinar si existe un conjunto H de tamaño al menos k tal que todo subconjunto de B la intersección entre $B_i$ y H no es vacía \\}
      }%
    }
  \end{minipage}
\end{center}

Entonces veamos la siguiente reducción:

$$
    \text{Vertex Cover} \leq_p \text{Hitting Set} 
$$

Sea $G$ un grafo, $V$ su conjunto de vertices y $E$ su conjunto de aristas, entonces se toma $A$ como el conjunto de vertices. Ahora, que sera $B$, en este caso se podria definir como el conjunto de subconjunto de los pares $(v_i, v_j)$ siempre y cuando exista una arista entre dichos vertices, aunque, esto seria simplemente el conjunto de las aristas de $G$

Habiendo establecido lo anterior, se puede observar que si existe un Hitting Set de tamaño k  para los conjuntos $A = V(G)$ y $B = E(G)$ entonces necesariamente existe un vertex cover de tamaño al menos k, por lo tanto, sabiendo que vertex cover es un problema NP-Completo se puede concluir finalmente que Hitting set también es NP-Completo 
