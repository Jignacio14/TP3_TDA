\section{Problema a solucionar}
\renewcommand{\labelenumii}{\arabic{enumi}.\arabic{enumii}}
\renewcommand{\labelenumiii}{\arabic{enumi}.\arabic{enumii}.\arabic{enumiii}}
\renewcommand{\labelenumiv}{\arabic{enumi}.\arabic{enumii}.\arabic{enumiii}.\arabic{enumiv}}

Luego de haber analizado a todos los rivales gracias a tu ayuda, Scaloni definió
un cronograma de entrenamiento. Tiene definido qué hacer para cada día de acá
al mundial que viene, e incluso más. Para hacerlo más simple, para los próximos $n$
días. El entrenamiento del día $i$ demanda una 
cantidad de esfuerzo $e_i$, y podemos decir que la ganancia que nos da
dicho entrenamiento es igual al esfuerzo. El entrenamiento 
que corresponde al día $i$ (así como su esfuerzo y ganancia) son inamovibles: 
el Chiqui Tapia alquiló las herramientas específicas para cada día, y la AFA 
está muy ocupada organizando el torneo de $2^{30}$ equipos del año que viene para 
andar moviendo cosas. Si la cantidad de energía que se tiene para un día $i$
es $j < e_i$, entonces la ganancia que se obtiene en ese caso es justamente $j$.
(si se tiene más energía que $e_i$, no es que se pueda usar más para tener más ganancia).

A su vez, los jugadores no son máquinas. La cantidad de energía que tienen disponible
para cada día va disminuyendo a medida que pasan los entrenamientos. Suponiendo
que los entrenamientos empiezan con los jugadores descansados, el primer
día luego de dicho descanso los jugadores tienen energía $s_1$. El segundo día
luego del descanso tienen energía $s_2$, etc... Para cada día
hay una cantidad de energía, y podemos decir que $s_1 \geq s_2 \geq ... \geq s_n$.
Scaloni puede decidir dejarlos descansar un día, haciendo que la energía "se renueve"
(es decir, el próximo entrenamiento lo harían con energía $s_1$ nuevamente,
siguiendo con $s_2$, etc...). Obviamente, si descansan, el entrenamiento de ese
día no se hace (y no se consigue ninguna ganancia).   

Scaloni no sabe bien cómo hacer para definir los días que deba entrenarse y los días
que convenga descansar de tal forma de tener la mayor ganancia posible (y tener
mayores probabilidades de ganar el mundial que viene), pero Menotti, 
exponente del juego bonito en Argentina, le recomendó usar Programación Dinámica
para resolver este problema. Nos está pidiendo ayuda para poder resolver este
problema. 

\subsection{Nota }

        Esta versión del informe contiene las correcciones solicitadas, los segmentos con titulos que contengan un "*" contienen correcciones 
        
\subsection{Algunas reglas a considerar}

\begin{enumerate}
    \item Se poseen dos conjuntos $E_i \text{ y } S_j$ donde corresponden al esfuerzo a realizar en un dia i y la energia disponible para un dia j, respectivamente
\item $\lvert E_i \lvert \ = \lvert S_j \lvert$ 
\item Se construye una función esfuerzo cuya representación matematica seria:
$
Esfuerzo(E_i, S_j) = \begin{cases}
  E_i & \text{si }  E_i < S_j\\
  S_j & \text{si }  E_i > S_j\\
  0 & \text{si }  j <= 0 \text{ o }  i <=0 \\
\end{cases}
$
\item Del enunciado sabemos que $S_{k+1} \le S_k$
\item Para un esfuerzo $E_i$ solo se podra agregar a la solucion si se incluye antes de un dia $S_j$ con $i \leq j$ 
\end{enumerate}