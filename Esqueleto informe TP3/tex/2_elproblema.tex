\section{El problema}
Scaloni ya está armando la lista de 43 jugadores que van a ir al mundial 2026. 
Hay mucha presión por parte de la prensa para bajar línea de cuál debería ser 
el 11 inicial. Lo de siempre. 
Algunos medios quieren que juegue Roncaglia, otros quieren que juegue 
Mateo Messi, y así. Cada medio tiene un subconjunto de
jugadores que quiere que jueguen. A Scaloni esto no le importa, no va a dejar
que la prensa lo condicione, pero tiene jugadores jóvenes a los que esto
puede afectarles. 

Justo hay un partido amistoso contra Burkina Faso la semana que viene. Oportunidad
ideal para poner un equipo que contente a todos, baje la presión y poder 
aislar al equipo. 

El problema es, ¿cómo elegir el conjunto de jugadores que jueguen ese partido 
(entre titulares y suplentes que vayan a entrar)? Además, Scaloni quiere poder
usar ese partido para probar cosas aparte. No puede gastar el amistoso
para contentar a un periodista mufa que habla mal de Messi, por ejemplo. 
Quiere definir el conjunto más pequeño de jugadores necesarios para contentarlos 
y poder seguir con la suya. Con elegir
un jugador que contente a cada periodista/medio, le es suficiente. 

Ante este problema, Bilardo se sentó con Scaloni para explicarle que en realidad 
este es un problema conocido (viejo zorro como es, ya se comió todas las operetas 
de prensa así que se conoce este problema de memoria). Se sirvió una copa de \textit{Gatorei} 
y le comentó:

Esto no es más que un caso particular del Hitting-Set Problem. El cual es: Dado un conjunto 
$A$ de $n$ elementos y $m$ subconjuntos $B_1, B_2, ..., B_m$ de $A$
($B_i \subseteq A \forall i$) , queremos el subconjunto $C \subseteq A$ de menor tamaño tal 
que $C$ tenga al menos un elemento de cada
$B_i$ (es decir, $C \cap B_i \neq \emptyset$). En nuestro caso, $A$ son los jugadores 
convocados, los $B_i$ son los deseos de la
prensa, y $C$ es el conjunto de jugadores que deberían jugar contra Burkina Faso 
si o si". 

Bueno, ahora con un poco más claridad en el tema, Scaloni necesita de nuestra 
ayuda para ver si obtener este subconjunto se puede hacer de forma eficiente 
(polinomial) o, si no queda otra, con qué alternativas contamos. 